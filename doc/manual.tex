% Project:     DNS resolver
% @file        doc/documentation.tex
% @date        14.11.2023
% 
% @brief Documentation file
%
% @author Adam Ližičiar <xlizic00@stud.fit.vutbr.cz>

\documentclass[a4paper, 11pt]{article}

\usepackage[slovak]{babel}
\usepackage[utf8]{inputenc}
\usepackage[left=2cm, top=3cm, text={17cm, 24cm}]{geometry}
\usepackage{times}
\usepackage{verbatim}
\usepackage{enumitem}
\usepackage{listings}
\usepackage{color}
\usepackage{graphicx}
\usepackage{tabularx}
\usepackage[unicode]{hyperref}
\hypersetup{
    colorlinks=true,
    linkcolor=black,
    urlcolor=blue, 
    citecolor=blue
}
\definecolor{mygreen}{rgb}{0,0.6,0}
\definecolor{mygray}{rgb}{0.5,0.5,0.5}
\definecolor{mymauve}{rgb}{0.58,0,0.82}
\lstset{
  backgroundcolor=\color{white},
  basicstyle=\footnotesize,
  breaklines=true,
  captionpos=b,
  commentstyle=\color{mygreen},
  escapeinside={\%*}{*)},
  keywordstyle=\color{blue},
  stringstyle=\color{mymauve},
  tabsize=4
}


\begin{document}

	%%%%% Titulná stránka %%%%%
	\begin{titlepage}
		\begin{center}
			\includegraphics[width=0.77\linewidth]{res/logo_FIT.pdf} \\

			\vspace{\stretch{0.382}}

			\scalebox{2}{\Huge{Manuál}} \\
			\LARGE{DNS Resolver} \\
			\vspace{\stretch{0.618}}
		\end{center}

		\begin{minipage}[b]{0.4 \textwidth}
			\raggedright
			{\Large \today}
		\end{minipage}
		\hfill
		\begin{minipage}[b]{0.6 \textwidth}
			\raggedleft
			\Large
			Adam Ližičiar (xlizic00)\\
		\end{minipage}		
	\end{titlepage}

	%%%%% Obsah %%%%%
	\pagenumbering{roman}
	\setcounter{page}{1}
	\tableofcontents
	\clearpage

	\pagenumbering{arabic}
	\setcounter{page}{1}
	
	%%%%% Úvod %%%%%
	\section{Úvod}
	DNS (Domain Name System) resolver je~kľúčovou súčasťou internetovej infraštruktúry, umožňujúcou prevod ľudom čitateľných doménových mien (ako je~napríklad \texttt{www.fit.vut.cz}) na~IP adresy, ktoré~ ~používané pre~sieťovú komunikáciu. Údaje ukladá do databázy na serveroch a vďaka tomu pomáha ľudom pristupovať k údajom pomocou názvov domén, a nie iba pomocou IP adries. Tento dokument poskytuje prehľad funkcií a~implementácie programu \texttt{dns}, ktorý bol navrhnutý na~zasielanie dotazov na~DNS servery a~na~zobrazenie prijatých odpovedí na~štandartnom výstupe.\newline

	\subsection{Charakteristika DNS}
	DNS je~hierarchický a~decentralizovaný systém, ktorý~umožňuje užívateľom na~internete nájsť webové stránky a~iné služby prostredníctvom ľudom čitateľných názvov, tzv.~doménových mien. DNS servery sú~zodpovedné za~prevod týchto názvov na~IP adresy (alebo naopak), ktoré sú~nevyhnutné pre~nadviazanie sieťového spojenia.\newline
	Packet DNS sa skladá z hlavičky, otázky, odpovedi, autorizovanej odpovedi a zvyšných (doplňujúcich) dát.\newline
	\texttt{Hlavička} obsahuje ID packetu, QR flagy (rozlišujú medzi otázkou a odpoveďou), TC, TD, RA, Z, OPCODE, QDCOUNT, ANCOUNT, NSCOUNT a ARCOUNT. Hlavička má vždy veľkosť 12B.\newline
	\texttt{Otázka} sa skladá z NAME (meno domény alebo IP adresa), TYPE (typ záznamu) a CLASS (typ komunikácie).\newline
	\texttt{Odpoveď} obsahuje odpoveď zo servera a skladá sa z NAME, TYPE, CLASS, TTL (čas platnosti odpovede), DL (dĺžka ďalšieho záznamu) a z dát odpovede.\newline
	\texttt{Autorizovaná odpoveď} je odpoveďou zo servera, ktorý je autorizovaný.\newline
	\texttt{Doplňujúce dáta} obsahujú dáta, ktoré aj keď nie sú priamou odpoveďou na otázku, tak môžu mať s ňou niečo spoločné.
	
	\subsection{Funkcie a účel programu DNS resolver}
	Program \texttt{dns} je nástroj navrhnutý na~zasielanie dotazov na~DNS servery a~analýzu prijatých odpovedí. Umožňuje používateľom vykonávať dotazy v~čitelné podobe a~poskytuje informácie o~odpovediach od~DNS serverov. Priamo v~programe je~implementované zostavenie a~odoslanie packetov, bez~potreby nástrojov. Program podporuje komunikáciu pomocou UDP.	

	%%%%% Vytvorenie spustiteľného súboru %%%%%
	\section{Vytvorenie spustiteľného súboru a~automatizácia úloh}
	Prebieha pomocou Makefile na~automatizáciu kompilácie, testovania a~generovania dokumentácie. 
	\begin{itemize}
		\item \texttt{make}: Kompiluje zdrojové súbory projektu a~pomocou kompilátora \texttt{g++} vytvára spustiteľný súbor \texttt{dns}.
		\item \texttt{make run}: Spustí skompilovaný program \texttt{dns} s~preddefinovanými argumentmi, čím~demonštruje jeho funkčnosť.
		\item \texttt{make test}: Spustí testovacie skripty definované v adresári \texttt{tests}, čím overuje správnosť implementácie programu.
		\item \texttt{make doc}: Generuje dokumentáciu projektu vo~formáte PDF pomocou nástroja \texttt{pdflatex}. Všetky súborové operácie sú~vykonávané v~rámci adresára \texttt{doc} a~výsledná dokumentácia je~presunutá do~objektového adresára \texttt{obj} pre~udržanie poriadku.
		\item \texttt{make clean}: Odstráni všetky objektové a~dočasné súbory, ktoré boli vytvorené počas kompilácie, čím~udržuje projektový adresár čistý.
		\item \texttt{make cleanall}: Rozširuje funkcionalitu \texttt{clean} o~odstránenie spustiteľného súboru \texttt{dns}, vygenerovanej dokumentácie v~PDF a~všetkých výstupov z~testov.
	\end{itemize}
	\subsection{Požiadavky na systém}
	Pre~úspešnú kompiláciu a~fungovanie programu je~potrebné zabezpečiť, aby bol program spustený v~operačnom systéme Linux.
	
	%%%%% Spúštanie programu %%%%%
	\section{Spúštanie programu}
	Použitie príkazu \texttt{dns} je~nasledovné:
	\begin{verbatim}
	dns [-h] [-r] [-x] [-6] -s server [-p port] adresa
	\end{verbatim}
	Poradie parametrov je~ľubovoľné, ak nie je nastavená premenná prostredia $POSIXLY_CORRECT$. Každý parameter má~špecifický význam:
	\begin{itemize}
		\item \texttt{-h}: Vypíše vysvetlenie použitia parametrov do~štandartného výstupu. Príklad výstupu je v prílohe \ref{subsection:res_A}
		\item \texttt{-r}: Požadovaná rekurzia (Recursion Desired = 1), inak bez~rekurzie.
		\item \texttt{-x}: Reverzný dotaz namiesto priameho.
		\item \texttt{-6}: Dotaz typu \texttt{AAAA} namiesto preddefinovaného \texttt{A}.
		\item \texttt{-s server}: IP~adresa alebo~doménové meno servera, kam sa má~odoslať dotaz.
		\item \texttt{-p port}: Číslo portu, na~ktorý sa má~odoslať dotaz, preddefinovaný je~53.
		\item \texttt{adresa}: Adresa na zistenie.
	\end{itemize}
	
	\subsection{Príklady výstupov}
	\noindent Odpoveď na príkaz \texttt{./dns -r -s kazi.fit.vutbr.cz www.fit.vut.cz}
	\begin{verbatim}
		Authoritative: Yes, Recursive: Yes, Truncated: No
		Question section (1)
		  www.fit.vut.cz., A, IN
		Answer section (1)
		  www.fit.vut.cz., A, IN, 14400, 147.229.9.26
		Authority section (0)
		Additional section (0)
	\end{verbatim}

	\noindent Odpoveď na príkaz \texttt{./dns -r -6 -s kazi.fit.vutbr.cz www.fit.vut.cz}
	\begin{verbatim}
		Authoritative: Yes, Recursive: Yes, Truncated: No
		Question section (1)
		  www.fit.vut.cz., AAAA, IN
		Answer section (1)
		  www.fit.vut.cz., AAAA, IN, 14400, 2001:67c:1220:809::93e5:91a
		Authority section (0)
		Additional section (0)
	\end{verbatim}

	\noindent Odpoveď na príkaz \texttt{./dns -x -s kazi.fit.vutbr.cz 147.229.9.26}
	\begin{verbatim}
		Authoritative: Yes, Recursive: No, Truncated: No
		Question section (1)
		  26.9.229.147.in-addr.arpa., PTR, IN
		Answer section (1)
		  26.9.229.147.in-addr.arpa., PTR, IN, 14400, www.fit.vut.cz.
		Authority section (4)
		  9.229.147.in-addr.arpa., NS, IN, 14400, rhino.cis.vutbr.cz.
		  9.229.147.in-addr.arpa., NS, IN, 14400, gate.feec.vutbr.cz.
		  9.229.147.in-addr.arpa., NS, IN, 14400, guta.fit.vutbr.cz.
		  9.229.147.in-addr.arpa., NS, IN, 14400, kazi.fit.vutbr.cz.
		Additional section (0)
	\end{verbatim}
	
	%%%%% Analýza a interpretácia výstupov %%%%%
	\section{Analýza a~interpretácia výstupov}
	Táto sekcia rozoberá, čo~možno zistiť vďaka použitiu programu \texttt{dns}.
	\subsection{Rekurzívne a~nerekurzívne dotazy}
	Umožňuje určiť, či sa~má vykonať rekurzívny (-r) alebo nerekurzívny dotaz. Rekurzívne dotazy žiadajú DNS server o~kompletnú odpoveď, zatiaľ čo~nerekurzívne vrátia len informácie, ktoré server už má.
	\subsection{Reverzné dotazy}
	Program umožňuje vykonávať reverzné dotazy (-x), vďaka čomu možno zistiť doménové meno priradené k~určitej IP adrese.
	\subsection{Typy dotazov AAAA a~A}
	Poskytuje možnosť vykonávať dotazy typu AAAA~(-6) pre~získanie IPv6 adries alebo typu A pre~IPv4 adresy, čo umožňuje flexibilitu v~získavaní sieťových informácií.
	\subsection{Výstupy a~ich interpretácia}
	Výstupy programu sú~prehľadné a informatívne, poskytujú podrobné údaje o~odpovediach DNS serverov. Umožňujú získať dôležité informácie vrátane autoritatívnosti odpovede, rekurzívnych vlastností a~detailov o~dotazovanej adrese.
	\subsection{Využitie výsledkov}
	Na~základe analýzy výsledkov možno získať užitočné informácie pre~sieťovú diagnostiku, administráciu a~plánovanie.

	%%%%% Riešenia problémov a chybové hlášky %%%%%
	\section{Riešenia problémov a chybové hlášky}
	Problémové stavy sú v programe riešené pomocou chybových kódov a výpisov chýb na štandartný chybový výstup.
	Príloha \ref{subsection:res_errCodes} obsahuje tabuľku chybových stavov a ich kódov.
	Trieda, ktorá obsluhuje chyby, má nasledujúci kód:
	\begin{lstlisting}[language=C++, caption={Kód triedy Error}]
// Class will print error message and exit program 
class Error {
public:
		
	// Constructor with an optional variable parameter
	Error(int errCode = ERR_UNDEFINED, std::string message = "undefined", std::string variable = "") {
		// If variable is provided, replace %s in the message with variable
		if (!variable.empty()) {
			size_t pos = message.find("%s");
			if (pos != std::string::npos) {
				message.replace(pos, 2, variable);
			}
		}
	
		// Print the error message
		std::cout << ANSI_RED;
		std::cerr << "ERROR: ";
		std::cout << ANSI_RESET;
		std::cerr << message + "!\n";
			
		exit(errCode);
	}
};
	\end{lstlisting}
		
	
	%%%%% Architektúra programu %%%%%
	\section{Architektúra programu}
	Aplikácia je navrhnutá s viacerými modulmi, vrátane hlavného vstupného bodu, tried a knižníc, čo umožňuje modularitu a ľahšiu údržbu kódu. Diagram tried programu je v prílohe \ref{subsection:res_classDiagram}.

	\subsection{Hlavný súbor}
	Súbor \texttt{main.cpp} slúži ako hlavný vstupný bod programu, kde sa inicializujú a spúšťajú triedy aplikácie.
	\begin{lstlisting}[language=C++, caption={Kód hlavného súboru}]
#include "libraries/main.hpp"

/*
 *  Main function
 */
int main(int argc, char *argv[]) {

	DnsRequestSender *request = new DnsRequestSender;
	DnsResponseReceiver *response = new DnsResponseReceiver;
	int dnsResponseSize;

	// Parse arguments, send packet and parse response
	Arguments *arg = Arguments::parse_arguments(argc, argv);    	// Parse command-line arguments into an 'Arguments' object
	char *dnsResponse = request->execute(arg, &dnsResponseSize);    // Send DNS request
	response->parse(dnsResponse, &dnsResponseSize);                 // Parse DNS answer

	// Free memory
	delete[] dnsResponse;
	return 0;
	
}

	\end{lstlisting}
	
	\subsection{Triedy}
	V rámci adresára \texttt{classes} sú implementované nasledujúce triedy:
	
	\begin{itemize}
		\item \textbf{Arguments}: Obsahuje implementáciu triedy \texttt{Arguments}, ktorá spracováva argumenty príkazového riadku.
		\item \textbf{DNS}:
		\begin{itemize}
			\item \texttt{DnsRequestSender}: Táto trieda je zodpovedná za odosielanie DNS požiadaviek a čaká na príchod odpovede.
			\item \texttt{DnsResponseReceiver}: Trieda na spracovanie DNS odpovedí.
		\end{itemize}
		\item \textbf{Error}: Trieda \texttt{Error} a jej hlavičkový súbor \texttt{Error.hpp} sa zaoberajú spracovaním a reportovaním chýb v programe.
	\end{itemize}
	
	\subsection{Knižnice}
	\begin{itemize}
		\item \textbf{$ansi\_colors.hpp$}: Obsahuje farby, ktoré sú použité na štandartnom výstupe.
		\item \textbf{$dns\_constants.hpp$}: DNS konštanty potrebné pre chod programu.
		\item \textbf{$dns\_structures.hpp$}: Obsahuje štruktúru \texttt{dnsHeaders},  \texttt{DnsQuestions}, \texttt{DnsAnswers} a \texttt{SoaHeaders}.
		\item \textbf{$dns\_constants.hpp$}: Importuje všetky kódy tried, knižníc a hlavičkových súborov.
	\end{itemize}
	
	\subsection{Manuál}
	V adresári \texttt{man} sa nachádza súbor \texttt{dns.1}, ktorý obsahuje manuál pre tento program.

	%%%%% Testovanie programu %%%%%
	\section{Testovanie programu}
	Testovanie prebieha pomocou overovania rozdielov medzi očakávaným výstupom programu a výstupom programu po spustený testov. Testy je možné spustiť pomocou \texttt{make test}.\newline
	Príklad výstupu \texttt{make test} je možné vidieť v prílohe \ref{subsection:res_tests}
	\begin{lstlisting}[language=BASH, caption={Ukážka kódu pre A testy}]
for (( i=1; i<=$numOfTests_A; i++ )); do
	read -r args < $root_dir/$sub_dir/$i.in
		
	# Using the variables in the paths
	./dns $args > "$root_dir/$sub_dir/$i.programOut"
	result=$(diff -b "$root_dir/$sub_dir/$i.expectedOut" "$root_dir/$sub_dir/$i.programOut")
		
	if [ -n "$result" ]; then
		echo -e "[ \033[0;31mERR\033[0m ] TEST $i: Unxpected result"
		echo "$result\n"
		counterError=$((counterError + 1))
	else
		echo -e "[ \033[0;32mOK\033[0m ] TEST $i: Correct result"
	fi
done
	\end{lstlisting}
	

	
	%%%%% Rozšírenia %%%%%
	\section{Rozšírenia}
	\begin{itemize}
		\item \textbf{Chybové kódy a detailný výpis chyby na STDERR}: Obsahuje jedinečný chybový kód pre každý typ chyby. Na štandartný chybový výstup vypíše opis chyby aj s ukážkou chybového kódu.
		\item \textbf{\texttt{man dns}}: V \texttt{src/man/dns.1} je kód pre manuál tohto programu.
		\item \textbf{Spracovanie viacerých typov RDATA}: Program dokáže okrem typov CNAME, A, AAAA spracovať aj NS, MD, MF, SOA, MB, MG, MR, NULL, WKS, PTR, HINFO, MINFO, MX, TXT a SRV.
		\item \textbf{Kontrola vstupných argumentov}: Konroluje, či vstupné argumenty majú správny formát (napr. či je argument, kde má byť IPv4 adresa naozaj IPv4 adresa, takisto aj pre IPv6,...)
	\end{itemize}
	


	%%%%% Referencie %%%%%
	\section{Referencie}
	Mockapetris, P. (1987). "DOMAIN NAMES - IMPLEMENTATION AND SPECIFICATION". Network Working Group, Request for Comments: 1035, ISI, November 1987, https://datatracker.ietf.org/doc/html/rfc1035\newline\newline
	Thomson, S., Huitema, C., Ksinant, V., Souissi, M. (2003). "DNS Extensions to Support IP Version 6". Network Working Group, Request for Comments: 3596, Cisco, Microsoft, 6WIND, AFNIC, October 2003,\newline https://datatracker.ietf.org/doc/html/rfc3596\newline\newline
	Cloudflare. (n.d.). "What is DNS? | How DNS works". https://www.cloudflare.com/learning/dns/what-is-dns/ \newline\newline

	%%%%% Prílohy %%%%%
	\clearpage
	\section{Prílohy}
	
	\subsection{Výpis \texttt{./dns -h} na~štandartnom výstupe}
	\label{subsection:res_A}
	\begin{figure}[ht]
		\centering
		\includegraphics[width=1 \linewidth]{res/A.png}

		\caption{Príklad výstupu \texttt{./dns -h} na~štandartnom výstupe. Príklad bol realizovaný na~serveri \mbox{merlin.fit.vutbr.cz}}
	\end{figure}

	\newpage
	\subsection{Tabuľka chybových kódov}
	\label{subsection:res_errCodes}
	\begin{table}[h]
		\centering
		\begin{tabularx}{\textwidth}{|X|l|}
		\hline
		\textbf{Názov chyby}                    & \textbf{Chybový kód} \\ \hline
		ERR\_NONE                              & 0                   \\ \hline
		\multicolumn{2}{|X|}{\textit{Build Errors}}                    \\ \hline
		ERR\_BUILD                             & 1                   \\ \hline
		\multicolumn{2}{|X|}{\textit{Argument Errors}}                 \\ \hline
		ERR\_ARG\_INVALID\_ARGUMENT            & 10                  \\ \hline
		ERR\_ARG\_UNUSUAL\_PARAMETER           & 11                  \\ \hline
		ERR\_ARG\_MISSING\_SERVER              & 12                  \\ \hline
		ERR\_ARG\_CAN\_NOT\_BE\_X\_AND\_IPV6   & 13                  \\ \hline
		ERR\_ARG\_IS\_NOT\_IPV4\_ADDRESS       & 14                  \\ \hline
		ERR\_ARG\_IS\_NOT\_IPV6\_ADDRESS       & 15                  \\ \hline
		ERR\_ARG\_IS\_NOT\_WEBSITE             & 16                  \\ \hline
		\multicolumn{2}{|X|}{\textit{DNS Request Sender Errors}}       \\ \hline
		ERR\_DNS\_S\_UNABLE\_LOAD\_SERVER      & 30                  \\ \hline
		ERR\_DNS\_S\_SOCKET\_WAS\_NOT\_CREATED & 31                  \\ \hline
		ERR\_DNS\_S\_DNS\_CONNECTION\_FAILED   & 32                  \\ \hline
		ERR\_DNS\_S\_UNABLE\_LOAD\_DNS\_SERVER & 33                  \\ \hline
		ERR\_DNS\_S\_IP\_ADDRESS\_NOT\_FOUND   & 34                  \\ \hline
		ERR\_DNS\_S\_RESPONSE\_FAILED          & 35                  \\ \hline
		\multicolumn{2}{|X|}{\textit{DNS Response Receiver Errors}}    \\ \hline
		ERR\_DNS\_T\_UNDEFINED                 & 70                  \\ \hline
		\multicolumn{2}{|X|}{\textit{Tests Errors}}                    \\ \hline
		ERR\_TESTS\_UNDEFINED                  & 80                  \\ \hline
		\multicolumn{2}{|X|}{\textit{Documentation Errors}}            \\ \hline
		ERR\_DOC\_UNDEFINED                    & 90                  \\ \hline
		\multicolumn{2}{|X|}{\textit{Other}}                           \\ \hline
		ERR\_UNDEFINED                         & 100                 \\ \hline
		\end{tabularx}
		\caption{Zoznam chybových stavov a ich chybových kódov}
		\label{tab:errors}
	\end{table}
		
	\newpage
	\subsection{Diagram tried programu}
	\label{subsection:res_classDiagram}
	\begin{figure}[ht]
		\centering
		\includegraphics[width=1 \linewidth]{res/classDiagram.png}

		\caption{Diagram tried}
	\end{figure}

	\newpage
	\subsection{Príklad výstupu \texttt{make test}}
	\label{subsection:res_tests}
	\begin{figure}[ht]
		\centering
		\includegraphics[width=1 \linewidth]{res/tests.png}

		\caption{Výstup \texttt{make test}}
	\end{figure}

\end{document}
