% Project:     DNS resolver
% @file        doc/documentation.tex
% @date        14.11.2023
% 
% @brief Documentation file
%
% @author Adam Ližičiar <xlizic00@stud.fit.vutbr.cz>

\documentclass[a4paper, 11pt]{article}

\usepackage[slovak]{babel}
\usepackage[utf8]{inputenc}
\usepackage[left=2cm, top=3cm, text={17cm, 24cm}]{geometry}
\usepackage{times}
\usepackage{verbatim}
\usepackage{enumitem}
\usepackage{graphicx}
\usepackage[unicode]{hyperref}
\hypersetup{
    colorlinks=true,
    linkcolor=black,
    urlcolor=blue, 
    citecolor=blue
}

\begin{document}

	%%%%% Titulná stránka %%%%%
	\begin{titlepage}
		\begin{center}
			\includegraphics[width=0.77\linewidth]{res/logo_FIT.pdf} \\

			\vspace{\stretch{0.382}}

			\scalebox{2}{\Huge{Manuál}} \\
			\LARGE{DNS Resolver} \\
			\vspace{\stretch{0.618}}
		\end{center}

		\begin{minipage}[b]{0.4 \textwidth}
			\raggedright
			{\Large \today}
		\end{minipage}
		\hfill
		\begin{minipage}[b]{0.6 \textwidth}
			\raggedleft
			\Large
			Adam Ližičiar (xlizic00)\\
		\end{minipage}		
	\end{titlepage}

	%%%%% Obsah %%%%%
	\pagenumbering{roman}
	\setcounter{page}{1}
	\tableofcontents
	\clearpage

	\pagenumbering{arabic}
	\setcounter{page}{1}
	
	%%%%% Úvod %%%%%
	\section{Úvod}
	//todo: Krátky prehľad o programe a jeho účele.
	//todo: Krátka charakteristika DNS
	//todo: Informácie o verzii a autorských právach.
	Napište program dns, který bude umět zasílat dotazy na DNS servery a v čitelné podobě vypisovat přijaté odpovědi na standardní výstup. Sestavení a analýza DNS paketů musí být implementována přímo v programu dns. Stačí uvažovat pouze komunikaci pomocí UDP.



	Není-li v jiné části zadání specifikováno jinak, je při vytváření programu povoleno použít hlavičkové soubory pro práci se sokety a další obvyklé funkce používané v síťovém prostředí (jako je netinet/*, sys/*, arpa/* apod.), knihovnu pro práci s vlákny (pthread), signály, časem, stejně jako standardní knihovnu jazyka C (varianty ISO/ANSI i POSIX), C++ a STL. Jiné knihovny nejsou povoleny.



	Spuštění aplikace

	Použití: dns [-r] [-x] [-6] -s server [-p port] adresa



	Pořadí parametrů je libovolné. Popis parametrů:

	-r: Požadována rekurze (Recursion Desired = 1), jinak bez rekurze.
	-x: Reverzní dotaz místo přímého.
	-6: Dotaz typu AAAA místo výchozího A.
	-s: IP adresa nebo doménové jméno serveru, kam se má zaslat dotaz.
	-p port: Číslo portu, na který se má poslat dotaz, výchozí 53.
	adresa: Dotazovaná adresa.
	Podporované typy dotazů



	Program dns se v odpovědích musí vypořádat se záznamy typu CNAME. Není požadována podpora DNSSEC.





	%%%%% Inštalácia a konfigurácia %%%%%
	\section{Inštalácia a konfigurácia}
	//todo: Pokyny pre inštaláciu programu.
	//todo: Požiadavky na systém a kompatibilita.
	
	%%%%% Spúštanie programu %%%%%
	\section{Spúštanie programu}
	//todo: Podrobný popis ako spustiť aplikáciu.
	//todo: Vysvetlenie rôznych spustiteľných parametrov (ako -r, -x, -6, -s, -p a iné).
	%//staci v dokumentacii: V zadání projektu je napsáno, že argumenty lze zadávat v libovolném pořadí. Avšak na serveru Eva je nastavena proměnná prostředí POSIXLY_CORRECT, což způsobí zastavení zpracování, pokud funkce getopt narazí na argument, který není obsažen v optstring. Z tohoto důvodu musí být dotazované jméno uvedeno jako poslední argument. Ostatní argumenty lze zadávat v libovolném pořadí. Je třeba něco změnit v implementaci, nebo stačí tuto podmínku zaznamenat v dokumentaci
	// co robi aky parameter.. aj help a vystup

	
	Výstup aplikace



	Na standardní výstup vypište informaci o tom, zda je získaná odpověď autoritativní, zda byla zjištěna rekurzivně a zda byla odpověď zkrácena. Dále vypište všechny sekce a záznamy obdržené v odpovědi.



	Pro každou sekci v odpovědi vypište její název a počet získaných záznamů. Pro každý záznam vypište jeho název, typ, třídu, TTL a data.



	Ukázka možného výstupu:



	%$ dns -r -s kazi.fit.vutbr.cz www.fit.vut.cz

	Authoritative: No, Recursive: Yes, Truncated: No

	Question section (1)

	www.fit.vut.cz., A, IN

	Answer section (1)

	www.fit.vut.cz., A, IN, 14400, 147.229.9.26

	Authority section (0)

	Additional section (0)

	%$ dns -r -s kazi.fit.vutbr.cz www.github.com

	Authoritative: No, Recursive: Yes, Truncated: No

	Question section (1)

	www.github.com., A, IN

	Answer section (2)

	www.github.com., CNAME, IN, 3600, github.com.

	github.com., A, IN, 60, 140.82.121.3

	Authority section (0)

	Additional section (0)


	je tam kontrola na format vstupu argumentov
	%%%%% Analýza a interpretácia výstupov %%%%%
	\section{Analýza a interpretácia výstupov}
	//todo: Ako čítať a interpretovať výstupy programu.
	//todo: Príklady výstupov a ich význam.

	%%%%% Riešenia problémov a chybové hlášky %%%%%
	\section{Riešenia problémov a chybové hlášky}
	//todo: Bežné problémy a ich riešenia.
	//todo: Vysvetlenie chybových hlášok a kroky k ich odstráneniu.
	
	%%%%% Architektúra programu %%%%%
	\section{Architektúra programu}
	//todo: Stručný popis kľúčových častí programu.

	%%%%% Testovanie programu %%%%%
	\section{Testovanie programu}
	//todo: Postupy a nástroje pre testovanie aplikácie.
	//todo: Príklady testov a ich význam.
	// testy su ok, a treba cakat kvoli zmene TTL.....Obsah databáze DNS se z principů mění a může vypadat z různých míst různě. Např. doporučuji se serveru kazi.fit.vutbr.cz ptát ze sítě FIT. Do jiných sítí mohou být odpovědi jiné. Ukázkové záznamy jsou mimo mou kontrolu a mohou se nepředvídatelně měnit, v průběhu řešení projektu mohou být zrušená, dočasně nedostupná, nově přesměrovaná jinam apod.
	
	Referenční prostředí pro překlad a testování

	Program by měl být přenositelný. Referenční prostředí pro překlad budou servery eva.fit.vutbr.cz a merlin.fit.vutbr.cz (program musí být přeložitelný a funkční na obou systémech). Vlastní testování může probíhat na jiném počítači s nainstalovaným OS GNU/Linux, či FreeBSD, včetně jiných architektur než Intel/AMD, jiných distribucí, jiných verzí knihoven apod. Pokud vyžadujete minimální verzi knihovny (dostupné na serveru merlin a eva), jasně tuto skutečnost označte v dokumentaci a README.

	%error testy testuju, ci na vystupe je chybovy vystup ""

	%%%%% Rozšírenia %%%%%
	\section{Rozšírenia}
	//todo: Aké rozšírenia má program? (podpora viac vecí, chybové kódy, chybové hlášky,...)
	// Errcodes
	// Zadání nespecifikuje, které typy RDATA bychom měli být schopni zpracovávat. Existuje více než 16 typů, ale zadání se zmiňuje pouze o CNAME, A, AAAA. Měl by je program umět zpracovat všechny, nebo jen tyto tři?:::::	Tato funkcionalita je vyžadovaná v zadání. Zpracování dalších, vámi uvedených i typů záznamů z jiných RFC, lze vnímat jako funkcionalitu základní zadání rozšiřující. Pakliže se rozhodnete implementovat funkcionalitu nad rámec zadání, jasně toto rozšíření označte v souboru README a dokumentaci. Vámi implementované rozšíření řádně dokumenujte. Vaše rozhodnutí ve fázi návrhu a implementace řádně dokumentujte a vysvětlete v dokumentaci.
	je tam kontrola na format vstupu argumentov
	// man stranka v src/man/dns.1


	%%%%% Referencie %%%%%
	\section{Referencie}
	//todo: referencie
	https://datatracker.ietf.org/doc/html/rfc1035
	https://datatracker.ietf.org/doc/html/rfc3596

	%%%%% Prílohy %%%%%
	\section{Prílohy}
	// wireshark photo


\end{document}
