% Project:     DNS resolver
% @file        doc/documentation.tex
% @date        14.11.2023
% 
% @brief Documentation file
%
% @author Adam Ližičiar <xlizic00@stud.fit.vutbr.cz>

\documentclass[a4paper, 11pt]{article}

\usepackage[slovak]{babel}
\usepackage[utf8]{inputenc}
\usepackage[left=2cm, top=3cm, text={17cm, 24cm}]{geometry}
\usepackage{times}
\usepackage{verbatim}
\usepackage{enumitem}
\usepackage{graphicx}
\usepackage[unicode]{hyperref}
\hypersetup{
    colorlinks=true,
    linkcolor=black,
    urlcolor=blue, 
    citecolor=blue
}

\begin{document}

	%%%%% Titulná stránka %%%%%
	\begin{titlepage}
		\begin{center}
			\includegraphics[width=0.77\linewidth]{res/logo_FIT.pdf} \\

			\vspace{\stretch{0.382}}

			\scalebox{2}{\Huge{Manuál}} \\
			\LARGE{DNS Resolver} \\
			\vspace{\stretch{0.618}}
		\end{center}

		\begin{minipage}[b]{0.4 \textwidth}
			\raggedright
			{\Large \today}
		\end{minipage}
		\hfill
		\begin{minipage}[b]{0.6 \textwidth}
			\raggedleft
			\Large
			Adam Ližičiar (xlizic00)\\
		\end{minipage}		
	\end{titlepage}

	%%%%% Obsah %%%%%
	\pagenumbering{roman}
	\setcounter{page}{1}
	\tableofcontents
	\clearpage

	\pagenumbering{arabic}
	\setcounter{page}{1}
	
	%%%%% Úvod %%%%%
	\section{Úvod}
	//todo: Krátky prehľad o programe a jeho účele.
	//todo: Krátka charakteristika DNS
	//todo: Informácie o verzii a autorských právach.

	%%%%% Inštalácia a konfigurácia %%%%%
	\section{Inštalácia a konfigurácia}
	//todo: Pokyny pre inštaláciu programu.
	//todo: Požiadavky na systém a kompatibilita.
	
	%%%%% Spúštanie programu %%%%%
	\section{Spúštanie programu}
	//todo: Podrobný popis ako spustiť aplikáciu.
	//todo: Vysvetlenie rôznych spustiteľných parametrov (ako -r, -x, -6, -s, -p a iné).
	//staci v dokumentacii: V zadání projektu je napsáno, že argumenty lze zadávat v libovolném pořadí. Avšak na serveru Eva je nastavena proměnná prostředí POSIXLY_CORRECT, což způsobí zastavení zpracování, pokud funkce getopt narazí na argument, který není obsažen v optstring. Z tohoto důvodu musí být dotazované jméno uvedeno jako poslední argument. Ostatní argumenty lze zadávat v libovolném pořadí. Je třeba něco změnit v implementaci, nebo stačí tuto podmínku zaznamenat v dokumentaci?
	// -x a -6 nemoze byt, pretoze zaroven nemoze byt PTR a AAAA

	%%%%% Analýza a interpretácia výstupov %%%%%
	\section{Analýza a interpretácia výstupov}
	//todo: Ako čítať a interpretovať výstupy programu.
	//todo: Príklady výstupov a ich význam.

	%%%%% Riešenia problémov a chybové hlášky %%%%%
	\section{Riešenia problémov a chybové hlášky}
	//todo: Bežné problémy a ich riešenia.
	//todo: Vysvetlenie chybových hlášok a kroky k ich odstráneniu.
	
	%%%%% Architektúra programu %%%%%
	\section{Architektúra programu}
	//todo: Stručný popis kľúčových častí programu.

	%%%%% Testovanie programu %%%%%
	\section{Testovanie programu}
	//todo: Postupy a nástroje pre testovanie aplikácie.
	//todo: Príklady testov a ich význam.

	%%%%% Rozšírenia %%%%%
	\section{Rozšírenia}
	//todo: Aké rozšírenia má program? (podpora viac vecí, chybové kódy, chybové hlášky,...)
	// Errcodes
	// Zadání nespecifikuje, které typy RDATA bychom měli být schopni zpracovávat. Existuje více než 16 typů, ale zadání se zmiňuje pouze o CNAME, A, AAAA. Měl by je program umět zpracovat všechny, nebo jen tyto tři?:::::	Tato funkcionalita je vyžadovaná v zadání. Zpracování dalších, vámi uvedených i typů záznamů z jiných RFC, lze vnímat jako funkcionalitu základní zadání rozšiřující. Pakliže se rozhodnete implementovat funkcionalitu nad rámec zadání, jasně toto rozšíření označte v souboru README a dokumentaci. Vámi implementované rozšíření řádně dokumenujte. Vaše rozhodnutí ve fázi návrhu a implementace řádně dokumentujte a vysvětlete v dokumentaci.


	%%%%% Referencie %%%%%
	\section{Referencie}
	//todo: referencie
	https://datatracker.ietf.org/doc/html/rfc1035
	https://datatracker.ietf.org/doc/html/rfc3596

	%%%%% Prílohy %%%%%
	\section{Prílohy}
	//todo: referencie


\end{document}
